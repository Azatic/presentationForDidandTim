% !TeX spellcheck = ru_RU
% !TEX root = vkr.tex

\label{sec:relatedworks}
% \emph{Обзор должен быть.} Здесь нужно писать, что индустрия и наука уже сделали по вашей теме. Он нужен, чтобы Вы случайно не изобрели какой-нибудь велосипед.

% По-английски называется related works или previous works.

% Если Ваша работа является развитием предыдущей и плохо понимаема без неё, то обзор должен идти в начале. Если же Вы решаете некоторую задачу новым интересным способом, то если поставить обзор в начале, то читатель может устать, пока доберется до вашего решения. В этом случае уместней поставить обзор в конце работы.

% Отечественным аналогом программы для составления расписания является программа:"1c.Автоматическое составление расписания".
% В данной программе составление расписания реализовано в несколких видах
В данном разделе будут рассмотрены готовые продукты для составления расписания.
% \subsection{Обзор методов}
% Задача составления расписания часто встречается в нашей жизни, поэтому ей было уделено немалое внимание со стороны научного сообщества. В данном разделе не будет проводится обзор ручного метода составления расписания, потому что данный метод теряет свою актуальность с каждым днем.
Обзор существующих методов решения задач теории расписания, а не продуктов, можно увидеть в работе~\cite{Прорасписание}.
  \subsection{Обзор решений}
Для автоматического составления расписания реализованы различные решения. В открытом доступе затруднительно найти информацию об алгоритмах составления расписания данных решений. В данном разделе рассмотрим продукты с точки зрения получения расписания. Данные о решениях собраны из различных открытых источников, основной из них ~\cite{Обзоррешение}.
  \begin{itemize}
    \item БИТ.ВУЗ.Расписание.\footnote{https://spb.1cbit.ru/1csoft/bit-vuz-raspisanie/}
    \item 1С:Автоматизированное составление расписания.Университет.\footnote{https://solutions.1c.ru/catalog/asp\_univer/materials}
    \item Экспресс-расписание Колледж.\footnote{https://pbprog.ru/docs/raspis/}
    \item Система «АВТОРасписание».\footnote{https://www.mmis.ru/programs/avtor}
    \item Расписание занятий: «Ректор-Колледж».\footnote{https://rector.spb.ru/}
  \end{itemize}
  Рассмотрим лидирующие в области решения:

\subsubsection{БИТ.ВУЗ.Расписание}
У решения есть полная и ``lite'' версии, рассмотрим полную:\\
% \begin{center}
  % \textbf{Бит.ВузРасписание}
   % \end{center}
  Преимущества данного решения:
  \begin{itemize}
    \item возможность изменять расписание с помощью таблиц;
    \item возможность учесть/игнорировать множество ограничений;
    \item мониторинг ошибок при составлении;
    \item функция составления расписания в смешанном режиме.
  \end{itemize}
  Недостатки:
  \begin{itemize}
      \item возможность составить расписание автоматически только для одной группы за раз;
      %\item Алгоритм составления расписания является жадным
      \item нет возможности использовать бесплатно.
  \end{itemize}
% Реализованные ограничения:
%   \begin{itemize}
%       \item Не более 4 пар в день (и тд, потом написать)
      
%   \end{itemize}

  \subsubsection{1C:Автоматизированное составление расписания}
    Преимущества данного решения:
  \begin{itemize}
    \item возможность изменять расписание с помощью таблиц;
    \item возможность выбора периодичности расписания (неделя,две недели и т.д.);
    \item возможность учесть/игнорировать множество ограничений;
    \item возможность установить ограничения по переходам между корпусами и аудиториями;
    \item мониторинг ошибок при составлении;
    \item функция составления расписания в смешанном режиме;
    \item удобный web-сервис.
  \end{itemize}
  Недостатки:
  \begin{itemize}
      %\item Возможность составить расписание автоматически только для одной группы
     % \item Алгоритм составления расписания является жадным
      \item нет возможности использовать бесплатно;
      \item до покупки затруднительно найти исчерпывающую информацию об алгоритме и о возможностях программы для автоматического построения расписания.
  \end{itemize}
  % Реализованные ограничения:
  % \begin{itemize}
  %     \item Не более 4 пар в день (и тд, потом написать)
  %     \end{itemize}

\textbf{Вывод}

Хотелось бы создать продукт с ясным алгоритмом составления расписания и возможностью составлять расписания на несколько потоков.
\subsection{Используемые инструменты}
\subsubsection{miniKanren} 
Для получения расписания используется miniKanren --- семейство встраеваемых языков реляционного программирования. 

Основные определения и понятия. Более подробную информацию по miniKanren можно увидеть в статье
~\cite{berd}

Программы на miniKanren пишутся с помощью реляций (от англ.``relation'' --- отношение).
Реляция --- вычислимое отношение. 
%miniKanren использует при поиске interliving search.
Ядром miniKanren в языке Racket являются три оператора: $cond^e$ $\equiv$ (введено вместо ==) $fresh$.

$\equiv$ называется унификацией и используется для того, чтобы показать, что два аргумента являются одинаковыми. 
\begin{lstlisting}[caption=Пример применения унификации, language=OCaml, frame=single]
(run 1 (q) ($\equiv$ q 5)) $\implies$ (5) 
(run 1 (q) ($\equiv$ q 5) ($\equiv$ q 6)) $\implies$ () 
\end{lstlisting}
В первой строке программа возвращает (5), но вторая возвращает пустой список, так как нет такого значения q, которое ровно одновременно и 5, и 6.

$cond^e$ зачастую используется для получения нескольких ответов. Рассмотрим применение $cond^e$ на примере.
\begin{lstlisting}[caption=Пример применения $cond^e$, language=OCaml, frame=single]
(run$\infty$ (q) (conde
           [(== q '(matan))];1 ветка
           [(== q '(alg))]; 2 ветка
           [(== q '(geom))]; 3 ветка
           )) $\implies$ ((matan) (alg) (geom))
\end{lstlisting}
Получен ответ ((matan) (alg) (geom)), это означает, что любой из трех списков удовлетворяет реляции $cond^e$. Реляция $cond^e$ ``удовлетворена'', точнее считается завершенной успешно, если успешна завершается хотя бы одна линия $cond^e$. Также стоит обратить внимание на порядок вывода ответов, $cond^e$ выполняет ветки по порядку их написания.

$fresh$ используется для создания свежих переменных
\begin{lstlisting}[caption=Использование $fresh$ и унификации, language=OCaml, frame=single]
(run 1 (q) (fresh (x y) ($\equiv$ x y) ($\equiv$ x 3) ($\equiv$ y q))) $\implies$ (3)
\end{lstlisting}
Таким образом, в данной программе x, y и q равны 3 и переменные перестают быть свежими, теперь попытка унификации любой из переменных с любым значение, кроме 3, приведет к пустому выводу программы.



%Особенностью miniKanren является то, что программы на нем пишутся с помощью реляций (от англ.``relation'' --- отношение).
%это функция, в которой нет различия между входными и выходными данными и которая "связывает", то есть задает некоторое отношение между переменными.


% Как упоминалось, наша задача принадлежит классу NP-полных задач.
% %( Алгоритма, для гарантированного поиска корректного расписания нет)!!!!! ОЧень сильное высказывание.
% MiniKanren, используя interliving search, имеет возможность найти ответы за разумное время.
%Лучшим вариантом для составления расписания является метод "ветвей и границ", который и использует miniKanren при поиске ответа.\\


Выбрана каноническая реализация miniKanren на Racket \\``faster-miniKanren''~\cite{faster}


\subsubsection{Racket\footnote{https://racket-lang.org/}}
Так как выбрана реализация miniKanren на Racket, и для оценки результата требуется лишь визуализация в виде общей таблицы, то было решено не включать дополнительных технологий в работу и использовать язык Racket для визуализации.


