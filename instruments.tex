\subsection{miniKanren} 
Для получения расписания по начальным данным использован miniKanren --- представитель семейства языков реляционного программирования. Особенностью miniKanren является то, что программы на нем пишутся с помощью реляций. Где реляция это функция, в которой нет различия между входными и выходными данными.


Как упоминалось, наша задача принадлежит классу NP-полных задач.
%( Алгоритма, для гарантированного поиска корректного расписания нет)!!!!! ОЧень сильное высказывание.
miniKanren, используя поиск в ширину, имеет возможность найти ответы за разумное время.
%Лучшим вариантом для составления расписания является метод "ветвей и границ", который и использует miniKanren при поиске ответа.\\


Выбрана каноническая реализация miniKanren на Racket ``faster-miniKanren''~\cite{faster}


\subsection{Racket}
Так как выбрана реализация miniKanren на Racket, и для оценки результата требуется лишь визуализация в виде общей таблицы, то было решено не включать дополнительных технологий в работу и использовать язык Racket для визуализации.
