% !TeX spellcheck = ru_RU
% !TEX root = vkr.tex


%цитировать поменьше и записывать 
%``Задачи теории расписаний связаны с построением расписаний, т.е. с упорядочиванием некоторых работ по времени и/или исполнителям с учетом ограничений''~\cite{Алгоритмы}.

В наше время до сих пор остается актуальной проблема составления расписания. В середине прошлого века начался бурный рост теории расписаний ~ \cite{Прорасписание}. Задачи теории расписаний связаны с упорядочиванием некоторых работ по времени и/или исполнителям с учетом ограничений ~\cite{Алгоритмы}.
%``Задачи теории расписаний, разумеется, связаны с построением расписаний, т.е. с упорядочиванием некоторых работ (операций) по
%времени и/или по исполнителям (приборам). При этом необходимо учитывать ограничения на последовательность выполнения работ, ограничения, связанные с исполнителями, и т.п.''~\cite{Алгоритмы}
Одной из задач теории расписаний является задача построения расписания для учебных заведений.

В наше время расписание часто составляется вручную. В предыдущем предложении и далее слово ``расписание'' будет использоваться в контексте расписания для высших учебных заведений. При составлении расписания следует учитывать ряд формальных ограничений, таких как, например, невозможность проводить одному преподавателю два занятия одновременно, невозможность проводить лекционное занятие в обычном классе и другие. Также при составлении расписания желательно учесть и множество других аспектов. Слишком большая кучность занятий увеличивает утомляемость учащихся. Следует учитывать пожелания преподавателей. % попытаться разделить на несколько предложений
% 2 и 3 абзац объединить

 Понятно, что учесть множество этих факторов при составлении расписания вручную можно, но это может быть сопряжено с некоторыми трудностями.Например, если диспетчер расписания на некотором этапе не сможет корректно поставить занятие, то небольшое перестроение расписания может привести к кардинальному изменению всего расписания, что усложняет работу.

%  Также работу диспетчера усложняют, точнее делают ее невозможной, неверные данные, то есть такие данные, по которым составить
% корректное расписание невозможно. Однако выявить, что эти данные
% неверны не так то просто.

% Также важно помнить, что часть задач составления расписания, к которой относится и наша задача, принадлежит классу NP-полных задач ~\cite{ТеорияРасписаний}, которые нельзя решить каким-либо известным полиномиальным алгоритмом.

% ``В настоящий момент широкое распространение имеют метаэвристические алгоритмы, которые находят “хорошее” решение, близкое к оптимальному, за приемлемое время. Недостатком таких алгоритмов является отсутствие оценок качества полученного решения. Неизвестно, насколько решение отличается от оптимального в наихудшем случае''
Задаче составления расписания было уделено внимание со стороны научного сообщества и были разработаны различные методы поиска решения. Учитывая ограничения, включая NP-полноту задачи, хорошо себя показывают эвристические методы, такие как имитация отжига, поиск с запретами, эволюционные алгоритмы ~\cite{Алгоритмы}. Также широкое распространение для решения задач теории расписаний получил метод программирования в ограничениях ~\cite{Алгоритмы}.
% ``При учете всех ограничений, включая NP-полноту задачи, на передний план выходят эвристические методы, такие как имитация отжига, поиск с запретами , эволюционные алгоритмы''~\cite{Алгоритмы}.

% ``В последнее время широкое распространение получил метод программирования в ограничениях (ПвО, в англоязычной литературе – Constraint
% Programming). Одной из областей его успешного применения является
% теория расписаний''~\cite{Прорасписание}. %попытаться сделать меньше ссылок
% % \footnote{Лазарев Александр Алексеевич
% % Гафаров Евгений Рашидович
% % ТЕОРИЯ РАСПИСАНИЙ
% % ЗАДАЧИ И АЛГОРИТМЫ}.
%``Широкое распространение получил метод программирования в ограничениях''~\cite{Прорасписание}. Он довольно успешно применяется для решения задач теории расписаний.

MiniKanren --- семейство встраеваемых языков реляционного программирования, который ранее не использовался при решении данной задачи, если судить по открытым источникам. MiniKanren является языком программирования в ограничениях, поэтому хотелось бы раскрыть потенциал полезного в академическом плане языка при решении практической задачи. В рамках данной работы предлагается разработать решение задачи построения расписания в вузах с помощью языка реляционного программирования miniKanren. 


\blfootnote{
	Дата сборки: \today\\
	
