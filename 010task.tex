% !TeX spellcheck = ru_RU
% !TEX root = vkr.tex

\label{sec:task}
 Целью работы является создание WEB-приложения с возможностью автоматического построения расписания занятий в вузах с помощью miniKanren со следующими начальными данными:
  \begin{itemize}
 \item  учебный план всех групп;
 \item  количество аудиторий и их специализация;
 \item  данные педагогической нагрузки;
 \item  набор ограничений.
 \end{itemize}
 
 Для её выполнения были поставлены следующие задачи:

 \textbf{Осенний семестр}
 \begin{enumerate}
 %\item Провести обзор аналогов
 \item Разработать процедуру поиска расписания, соответствующего следующим ограничениям:
 \begin{itemize}
         \item учет всех предметов из учебного плана;
 %   \item Вставка не более 5 пар в день
    \item проверка аудиторий на специализацию.
 \end{itemize}
 \item Визуализировать ответ в виде одной общей таблицы с расписанием.
 \end{enumerate}

 \textbf{Весенний семестр}
 \begin{enumerate}
    \item Добавить оценку расписания.
     \item Добавить ограничения:
     \begin{itemize}
         \item полное исключение окон;
    \item нельзя ставить одинокую пару;
    \item проверка аудиторий на вместимость.
 \end{itemize}
    \item Разработать WEB-приложение с возможностью доставать таблицы различных видов и забивать данные не в код.
 \end{enumerate}
